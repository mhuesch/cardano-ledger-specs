\section{Transactions}
\label{sec:transactions}

Transactions are defined in Figure~\ref{fig:defs:utxo-shelley}.
A transaction body, $\TxBody$, is made up of seven pieces:

\begin{itemize}
  \item A set of transaction inputs.
    This derived type identifies an output from a previous transaction.
    It consists of a transaction id and an index to uniquely identify the output.
  \item An indexed collection of transaction outputs.
    The $\TxOut$ type is an address paired with a coin value.
  \item A list of certificates, which will be explained in detail in
    Section~\ref{sec:delegation-shelley}.
  \item A transaction fee. This value will be added to the fee pot and eventually handed out
    as stake rewards.
  \item A time to live. A transaction will be deemed invalid if processed after this slot.
  \item A mapping of reward account withdrawals.  The type $\Wdrl$ is a finite map that maps
    a reward address to the coin value to be withdrawn. The coin value must be equal
    to the full value contained in the account. Explicitly stating these values ensures
    that error messages can be precise about why a transaction is invalid.
    For reward calculation rules, see Section \ref{sec:reward-overview},
    and for the rule for collecting rewards, see Section \ref{sec:utxo-trans}.
  \item Update proposals for protocol parameters and software.
\end{itemize}

In the derived types, $\Metadata$ and $\Applications$ are values
that contain versions of current and next versions of applications.
The $\ApName$ uniquely identifies a specific kind of application (e.g.
the wallet), and the associated $\ApVer$ is the specific version of that
application. The associated $\Metadata$ value gives the \textit{next possible}
versions of the given application. It is a mapping of system tags to hashes of
installer binaries, and is needed for the update mechanism.
The update proposal type $\Update$ is a pair of

\begin{itemize}
  \item $\PPUpdate$, the protocol parameter upates
  \item $\AVUpdate$, the updates to Shelley software (applications)
\end{itemize}

The $\PPUpdate$ has to do with changing rules and constants of the ledger
protocol. The relationship between the rule update procedure and updating the
node software that implements these rules is outlined in the next section
(see Section \ref{sec:update}).

Note that both of these finite maps are indexed by the hashes of the keys of
entities proposing the given
updates, $\KeyHashGen$.
We use the abstract type $\KeyHashGen$ to represent hashes of genesis
(public verification) keys, which have type $\VKeyGen$.
Genesis keys are the keys belonging to the federated
nodes running the Cardano system currently (also referred to as core nodes).
The the regular user verification keys are of a type $\VKey$, distinct from the
genesis key type, $\VKeyGen$. Similarly, the type hashes of these
are distinct, $\KeyHash$ and $\KeyHashGen$ respectively.

Currently, updates
can only be proposed and voted on by the owners of the genesis keys.
The process of decentralization will result in the core nodes gradually giving up
some of their priviledges and responsibilities to the other system nodes.
The aim is for these nodes to eventually give them \textit{all} up.
The intent is that in the future, the update proposal mechanism will
be decentralized as well, and allow all nodes to participate (but likely
not as a feature in the Shelley release).
For more on the decentralization process,
see \ref{sec:new-epoch-trans}.

A transaction, $\Tx$, consists of:

\begin{itemize}
  \item The transaction body.
  \item A collection of witnesses, represented as a finite map from payment verification keys
    to signatures.
\end{itemize}

Additionally, the $\UTxO$ type will be used by the ledger state to store all the
unspent transaction outputs. It is a finite map from transaction inputs
to transaction outputs that are available to be spent.

Finally, $\fun{txid}$ computes the transaction id of a given transaction.
This function must produce a unique id for each unique transaction.

\begin{figure*}[htb]
  \emph{Abstract types}
  %
  \begin{equation*}
    \begin{array}{r@{~\in~}lr}
      \var{gkey} & \VKeyGen & \text{genesis public keys}\\
      \var{gkh} & \KeyHashGen & \text{genesis key hash}\\
      \var{txid} & \TxId & \text{transaction id}\\
      \var{an} & \ApName & \text{application name}\\
      \var{st} & \SystemTag & \text{system tag}\\
      \var{ih} & \InstallerHash & \text{update data}\\
      \var{exunits} & \ExUnits & \text{units of script execution} \\
      \var{isf} & \ForFee & \text{tag for inputs intended for script fees} \\
      \var{isf} & \NForFee & \text{tag for inputs not for script fees} \\
    \end{array}
  \end{equation*}
  \emph{Script types}
  %
  \begin{equation*}
    \begin{array}{r@{~\in~}l@{\qquad=\qquad}lr}
      \var{isf} & \IsFee & \ForFee \uniondistinct \NForFee \\
      \var{dat}
      & \Data
      & \mathbb{N}\uniondistinct\mathbb{H}\uniondistinct(\mathbb{N}\times\seqof{\Data})
        \uniondistinct\seqof{\Data}\uniondistinct\seqof{\Data \times \Data}
      & \text{the $\Data$ type}\\
    \end{array}
  \end{equation*}
  %
  \begin{equation*}
    \begin{array}{r@{~\in~}l@{\subset}lr}
      \var{script_v}&\ScriptV & \Script & \text{validator script}\\
      \var{script_d}&\ScriptD & \Data & \text{data script}\\
      \var{script_r}&\ScriptR & \Data & \text{redeemer script}\\
    \end{array}
  \end{equation*}
%
  \emph{Derived types}
  %
  \begin{equation*}
    \begin{array}{r@{~\in~}l@{\qquad=\qquad}lr}
      (\var{txid}, \var{ix})
      & \type{TxIn}
      & \TxId \times \Ix
      & \text{UTxO entry reference}
      \\
      \var{(\var{script_v}, \var{script_r},\var{exunits})}
      & \Info
      & \ScriptV \times \ScriptR \times \ExUnits
      & \text{validation data}
      \\
      (\var{txid}, \var{ix}, \var{inf})
      & \type{TxInScr}
      & \TxIn \times \Info
      & \text{script input}
      \\
      \var{txin}
      & \TxInTx
      & (\TxIn \times \IsFee) \uniondistinct \TxInScr
      & \text{transaction input}
      \\
      \var{ptx}
      & \PendingTxInS
      & \TxIn \times (\HashScr \times \HashScr) \times \Coin
      & \text{pending Tx script input}
      \\
      (\var{addr}, c)
      & \type{TxOutVK}
      & \Addr \times \Coin
      & \text{vk address output}
      \\
      (\var{addr}, c)
      & \type{TxOutScr}
      & \type{TxOutVK} \times \ScriptD
      & \text{script address output}
      \\
      \var{txout}
      & \TxOut
      & \TxOutVK \uniondistinct \TxOutScr
      & \text{transaction outputs}
      \\
      \var{utxo}
      & \UTxO
      & \TxIn \mapsto \TxOut
      & \text{unspent tx outputs}
      \\
      \var{wdrl}
      & \Wdrl
      & \AddrRWD \mapsto \Coin
      & \text{reward withdrawal}
      \\
      \var{pup}
      & \PPUpdate
      & \KeyHashGen \mapsto \Ppm \mapsto \Seed
      & \text{PP update}
      \\
      \var{av}
      & \ApVer
      & \N
      & \text{application versions}
      \\
      \var{md}
      & \Metadata
      & \SystemTag\mapsto\InstallerHash
      & \text{application metadata}
      \\
      \var{apps}
      & \Applications
      & \ApName \mapsto (\ApVer \times \Metadata)
      & \text{application versions}
      \\
      \var{aup}
      & \AVUpdate
      & \KeyHashGen \mapsto \Applications
      & \text{application update}
      \\
      \var{up}
      & \Update
      & \PPUpdate \times \AVUpdate
      & \text{update proposal}
    \end{array}
  \end{equation*}
  \caption{Definitions used in the UTxO transition system}
  \label{fig:defs:utxo-shelley-1}
\end{figure*}

\begin{figure*}[htb]
  \emph{PendingTx Types}
  %
  \begin{equation*}
    \begin{array}{r@{~\in~}l@{\qquad=\qquad}l}
      \var{ptxiv}
      & \PendingTxInV
      & \TxIn \times \Coin
      \\
      \var{ptxis}
      & \PendingTxInS
      & \TxIn \times (\HashScr \times \HashDat) \times \Coin
      \\
      \var{ptxi}
      & \PendingTxIn
      & \PendingTxInV \uniondistinct \PendingTxInS
      \\
      \var{ptx}
      & \PendingTx
      & \powerset{\PendingTxIn} \times  \powerset{\TxOut} \times \Coin \times
      \PendingTxInS \times (\Slot\times\Slot)
      \\
    \end{array}
  \end{equation*}
  \caption{Definitions used to make PendingTx}
  \label{fig:defs:utxo-pending}
\end{figure*}

\begin{figure*}[htb]
  \emph{Transaction Types}
  %
  \begin{equation*}
    \begin{array}{r@{~\in~}l@{\qquad=\qquad}l}
      \var{txbody}
      & \TxBody
      & \powerset{\TxInTx} \times (\Ix \mapsto \TxOut) \times \seqof{\DCert}
        \times \Coin \times (\Slot\times\Slot) \times \Wdrl \times \Update
      \\
      \var{wit} & \TxWitness & (\VKey \mapsto \Sig)
      \\
      \var{tx}
      & \Tx
      & \TxBody \times \TxWitness
    \end{array}
  \end{equation*}
  %
  \emph{Accessor Functions}
  \begin{equation*}
    \begin{array}{r@{~\in~}lr}
      \fun{txinputs} & \Tx \to \powerset{\TxInTx} & \text{transaction inputs} \\
      \fun{txouts} & \Tx \to (\Ix \mapsto \TxOut) & \text{transaction outputs} \\
      \fun{txcerts} & \Tx \to \seqof{\DCert} & \text{delegation certificates} \\
      \fun{txfee} & \Tx \to \Coin & \text{transaction fee} \\
      \fun{txsize} & \Tx \to \Coin & \text{transaction size} \\
      \fun{txttl} & \Tx \to \Slot & \text{time to live} \\
      \fun{txlst} & \Tx \to \Slot & \text{start of liveness interval} \\
      \fun{txwdrls} & \Tx \to \Wdrl & \text{withdrawals} \\
      \fun{txbody} & \Tx \to \TxBody & \text{transaction body}\\
      \fun{txwitsVKey} & \Tx \to (\VKey \mapsto \Sig) & \text{VKey witnesses} \\
      \fun{txup} & \Tx \to \Update & \text{protocol parameter update} \\
      \fun{scrsize} & \ScriptD \to \Coin & \text{data size} \\
      \fun{val} & \TxOut \to \Coin & \text{output value} \\
    \end{array}
  \end{equation*}
  \caption{Definitions used in the UTxO transition system, cont.}
  \label{fig:defs:utxo-shelley-2}
\end{figure*}

\begin{figure*}[htb]
  \emph{Abstract Functions}
  %
  \begin{align*}
    \fun{hashScript} \in & \Script\to \HashScr & \text{compute script hash} \\
    \fun{hashData} \in & \Data \to \HashDat
    & \text{compute hash of data} \\
    \fun{toData} \in & \PendingTx\to \Data
    & \text{encode PendingTx as Data} \\
  \end{align*}
  \emph{Helper Functions}
  %
  \begin{align*}
    & \fun{valScriptUpTo} \in \Script\to (\Data \times \Data \times \Data \times \N \times \N)
    \to\Bool \\ & \text{restricted validate script} \\~\\
    & \fun{pendingTxIn} \in \TxInTx \to \Coin \to \PendingTxIn \\
    & \fun{pendingTxIn}~\var{txin}~\var{c} = \\
    & \begin{cases}
          (txin,\fun{hashScript}~{script_v},\fun{hashData}~{script_v},c)
           & \text{if}~\var{txin} = ((txid,ix),(\var{script_v}, \var{script_r},\wcard)) \in \TxInScr \\
          ((txid,ix),c) & \text{if}
           \var{txin} = ((txid,ix),\wcard) \in \TxIn \times \IsFee
      \end{cases}\\
    & \text{build pending Tx input} \\~\\
    & \fun{mkIns} \in \Tx \to \UTxO \to \powerset{\PendingTxIn} \\
    & \fun{mkIns}~\var{tx}~\var{utxo} = \\
    & \{ \fun{pendingTxIn}~\var{txin}~\var{\fun{val}~\var{txout}} ~\vert~ \var{txin} \mapsto \var{txout}
    \in \fun{txins}~\var{tx} \restrictdom \var{utxo}\} \\
    & \text{build pending Tx inputs} \\~\\
    & \fun{validationData} \in \UTxO \to \Tx \to \PendingTxInS \to \PendingTx \\
    & \fun{validationData}~\var{utxo}~\var{tx}~\var{ptxis} = \\ &
    ((\fun{mkIns}~\var{tx}~\var{utxo}),\{ \var{txout}~\vert~ \var{ix} \mapsto \var{txout} \in \fun{txouts}~\var{tx}\},
    \fun{txfee}~{tx},\var{ptxis},(\fun{txlst}~{tx},\fun{txttl}~{tx})) \\
    & \text{build PendingTx} \\
  \end{align*}
  %
  \emph{Notation}
  %
  \begin{align*}
    \llbracket \var{script_v} \rrbracket_{\var{exunits}} \var{script_d}~\var{script_r}~\var{ptx}
    &=& \fun{valScriptUpTo} ~\var{script_v} ~ \var{script_d}~\var{script_r}~\var{ptx}~
    \var{exunits}
  \end{align*}
  \caption{Script Validation}
  \label{fig:defs:functions-valid}
\end{figure*}

Figure~\ref{fig:defs:functions-txins} shows the helper functions
$\fun{txinsVKey}$ and $\fun{txinsScript}$ which partition the set of transaction
inputs of the transaction into those that are locked with a private key and
those that are locked via a script.

\clearpage
